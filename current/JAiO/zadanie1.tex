%!TEX TS-program = xelatex
%!TEX encoding = UTF-8 Unicode


\title{Języki, Automaty i Obliczenia}
\author{
	Michał Balcerzak \\
	Wydział Matematyki, Informatyki i Mechaniki\\
	Uniwersytet Warszawski, \underline{Polska}\\
}
\date{\today}

\documentclass[12pt]{article}
\usepackage{amsmath}
\usepackage{graphicx, amsfonts}
\usepackage{scalefnt}
\usepackage{mathtools}
\usepackage{fancyhdr}
\usepackage{amsbsy}
\usepackage{listings}
\usepackage{amssymb}


\lstdefinestyle{mystyle}{
    stringstyle=\color{codepurple},
    basicstyle=\footnotesize\ttfamily,
    breakatwhitespace=false,         
    breaklines=true,                 
    captionpos=b,                    
    keepspaces=true,                 
    numbers=left,                    
    showspaces=false,                
    showstringspaces=false,
    showtabs=false,                  
}
 
\lstset{language=Python, style=mystyle}

\newcommand{\pagesss}{\pageref{LastPage}}

\pagestyle{fancy}
\lhead{\scriptsize Języki, Automaty i Obliczenia}
\chead{\textsf{ZADANIE 1}}
\rhead{\scriptsize Strona \thepage\ z \pagesss}

\cfoot{\footnotesize Strona \thepage\ z \pagesss}


\fancypagestyle{plain}{%
    \renewcommand{\headrulewidth}{0pt}%
    \fancyhf{}%
    \fancyfoot[C]{\footnotesize Strona \thepage\ z \pagesss}%
}

\tolerance=1
\emergencystretch=\maxdimen
\hyphenpenalty=10000
\hbadness=10000

\setcounter{secnumdepth}{0}


\newcommand*{\Autom}{\mathcal{A}}
\newcommand{\bigexists}{%
	\mathop{\lower0.75ex\hbox{%
			   \scalebox{1.7}{\ensuremath{\exists}}}}\limits}

\newcommand{\bigany}{%
	\mathop{\lower0.75ex\hbox{%
			   \scalebox{1.7}{\ensuremath{\forall}}}}\limits}

\newcommand{\bigsum}{%
	\mathop{\lower0.75ex\hbox{%
			   \scalebox{1.7}{\ensuremath{\Sigma}}}}\limits}

\newcommand*{\QEDB}{\hfill\ensuremath{\square}}%
\newcommand{\code}[1][]{{\ttfamily{#1}}}


\begin{document}
\maketitle

\section{Treść zadania}
Powiemy, że automat niedeterministyczny $\Autom$ jest \textit{k-niedeterministyczny}, gdzie
$k \in \mathbb{N}$, jeżeli dla każdego słowa wejściowego $w$, automat $\Autom$ ma co najwyżej
$k$ biegów po słowie $w$ zaczynających się w stanie początkowym.

Pokazać, że istnieje algorytm, który otrzymawszy $10$-niedeterministyczny automat $\Autom$,
w czasie wielomianowym produkuje równoważny mu automat deterministyczny

\paragraph{}
Dokładniej, czas działania algorytmu dla danego $10$-niedeterministycznego automatu $\Autom$
ma być ograniczony przez $poly(|Q|+|A|+|\delta|)$, gdzie ${poly:\mathbb{N}\rightarrow\mathbb{N}}$
jest pewnym wielomianem, oraz $Q,A,\delta$ to, odpowiednio: zbiór stanów, alfabet i zbiór
tranzycji automatu $\Autom$. Nie jest sprecyzowane jak algorytm ma działać, gdy wejściowy
automat $\Autom$ nie jest $10$-niedeterministyczny (może wtedy np. działać w czasie
wykładniczym, produkować nierównoważny automat, lub zwracać błąd).

\paragraph{}
Z rozwiązania powinno jasno wynikać istnienie takiego algorytmu. W szczególności, należy dokładnie
uzasadnić poprawność algorytmu i jego czas działania. Algorytm można opisać pseudokodem lub
słowami, lub jego istnienie może być wnioskiem z postaci przeprowadzonej konstrukcji matematycznej.

\clearpage

\section{Rozwiązanie}
Bieg $b_w$ automatu $10$-niedeterministycznego $\Autom$ po pewnym słowie
$w = a_1 a_2 a_3 \dots a_n$ to ciąg tranzycji zawartych w automacie $\Autom$:
$$b_{w,0} \xrightarrow{a_1} b_{w,1} \xrightarrow{a_2} b_{w,2} \xrightarrow{a_3}
\dots \xrightarrow{a_n} b_{w,n}$$
gdzie $b_{w,0}, b_{w,1}, \dots, b_{w,n} \in Q$.

\paragraph{}
Powiemy, że dwa biegi automatu $\Autom$ po pewnym słowie $w$ – {$b_w$ oraz $c_w$} – są
\textit{różne} wtedy i tylko wtedy gdy $\exists_{i \in \mathbb{N}_{|w + 1|}} b_{w,i} \neq c_{w,i}$
(gdzie $\mathbb{N}_{i} = \{0, 1, \dots, i - 1\}$). 

Z kolei przez to, że $\Autom$ ma $k$ biegów po słowie $w$ będziemy rozumieć to, że każde dwa biegi
spośród biegów $\Autom$ po słowie $w$ są różne.

\paragraph{}
\textit{Stanem końcowym} biegu $b_w$ nazwiemy $b_{w,|w|}$.


\paragraph{Wniosek 1:}
Po wczytaniu pewnego słowa $w \in A^*$ do $\Autom$ możemy zakończyć bieg w co najwyżej
$10$ różnych stanach.
\subparagraph{Dowód:}
W przeciwnym przypadku musiałoby istnieć przynajmniej $11$ biegów $\Autom$ po 
słowie $w$, spośród których każdy musiałby mieć inny stan końcowy. To jednak nie jest możliwe, 
gdyż wtedy każdy z tych biegów byłby różny od dowolnego z pozostałych, więc $\Autom$ musiałby mieć
co najmniej $11$ biegów po słowie $w$, co byłoby sprzeczne z jego własnością
$10$-niedeterministyczności. \QEDB

\paragraph{}
Przyjrzyjmy się procesowi determinizacji automatów niezdeterminizowanych przytoczonemu na
wykładzie. Dla przypomnienia – automat deterministyczny $(A, Q', q_0, F', \delta')$ otrzymamy
z automatu niedetermnistycznego $(A, Q, I_\Autom, F_\Autom, \delta)$ biorąc:
\begin{itemize}
  \item $Q' = P(Q)$ (zbiór potęgowy $Q$)
  \item $q_0 = I_\Autom$
  \item $F' = \{S \mid S \subseteq Q' \land S \cap F_\Autom \neq \emptyset\}$
  \item $\delta' = f \in Q'^{(Q'\times A)}$ takie, że
	$f(S, a) = \{q \in Q \mid \exists_{p \in S} (p, a, q) \in \delta\}$
\end{itemize}

\paragraph{Fakcik 1:}
Skrypt wykładowy mówi, że dla dowolnego $w \in A^*$ stan $q_0 \cdot w$ jest równy zbiorowi tych
stanów, które można osiągnąć w automacie $\Autom$ zaczynając w którymś ze stanów początkowych i
wczytując słowo $w$.

\paragraph{}
Niech automat $\Autom' = (A, Q', q_0, F', \delta')$ stanowi bezpośrednią determinizację
automatu $\Autom = (A, Q, I_\Autom, F_\Autom, \delta)$. 

\paragraph{Lemat 1:} Dla każdego osiągalnego stanu $q_{\Autom'}$ automatu $\Autom'$ zachodzi
${|q_{\Autom'}| \leq 10}$.
\subparagraph{Dowód:} 
Dowolny stan $q_{\Autom'} \in Q'$ jest osiągalny wtedy i tylko wtedy, gdy istnieje $w \in A^*$ że
$q_0 \cdot w = q_{\Autom'}$. Na podstawie fakciku 1 każdy osiągalny stan $q_{\Autom'}$ jest więc
tożsamy z pewnym zbiorem $T$ stanów osiągalnych poprzez wczytanie pewnego $w$ do $\Autom$.
Z kolei na podstawie wniosku 1 możemy stwierdzić, że zbiór $T$ musi mieć moc niewiększą niż $10$
- a więc moc tegoż dowolnie wybranego, osiągalnego stanu $q_{\Autom'}$ również ma takąż własność,
co kończy dowód. \QEDB

\paragraph{}
Automaty są równoważne gdy akceptują ten sam język. Na mocy wykładu $\Autom'$ jest równoważny
automatowi $\Autom$. Stwórzmy na podstawie $\Autom'$ automat $\mathcal{B}$ będący, opisując
intuicyjnie, tożsamy z tą częścią $\Autom'$, której stany mają moce niewiększe niż $10$. Automat
$\mathcal{B} = (A, \breve Q, q_0, \breve F, \breve\delta)$ zdefiniujemy następująco:
\begin{itemize}
  \item $\breve Q = \{S \mid S \in Q' \land |S| \leq 10\} \cup \{\bot\}$
  \item $\breve F = F' \cap \breve Q$
  \item $\breve\delta = f \in \breve Q ^{(\breve Q \times A)}$ takie, że
	$$f(S, a) = \begin{cases}
	    \delta'(S, a), & \text{gdy}\ S \neq \bot \land |\delta'(S, a)| \leq 10 \\
	    \bot, & \text{w przeciwnym przypadku}
        \end{cases}$$
\end{itemize}

\paragraph{}
Niech napis $\mathcal{C}(w)$ oznacza dla automatu $\mathcal{C}$ i słowa $w \in A^*$ stan osiągany
po wczytaniu $w$ przez automat $\mathcal{C}$.

\clearpage 

\paragraph{Lemat 2:}
Automat $\mathcal{B}$ jest równoważny automatowi $\Autom$, czyli
$${\bigany_{w \in A^*} \mathcal{B}(w) \in \breve F \Leftrightarrow \Autom(w) \in F_\Autom}$$
\subparagraph{Dowód:}
Weźmy dowolne $a_1 a_2 a_3 \dots a_n = w \in A^*$. Wiemy że
${\Autom(w) \in F_\Autom \Leftrightarrow \Autom'(w) \in F'}$. Wówczas:

\begin{itemize}
  \item $\mathcal{B}(\varepsilon) = \Autom'(\varepsilon) = q_0$
  \item Jeśli dla pewnego $v$ – $i$-literowego prefiksu $w$ gdzie $i \in \{0, 1, \dots, n-1\}$ –
	  zachodzi $\Autom'(v) = \mathcal{B}(v) = q$, to
	  $\breve\delta(q, a_{i+1}) = \delta'(q, a_{i+1})$, gdyż
  \begin{itemize}
    \item $q \neq \bot$, bo $q = \Autom'(v) \notin Q'$
    \item $\delta'(q, a_{i+1})$ jest osiągalny w $\Autom'$, bo $q$ jest osiągalny, a więc na mocy 
	    lematu 1 $|\delta'(q, a_{i+1})| \leq 10$
  \end{itemize}
  Stąd, na mocy indukcji wynika $\mathcal{B}(w) = \Autom'(w)$. Ponadto stan $\Autom'(w)$ jest
  osiągalny, więc z własności $\mathcal{B}$ należy również do $\breve Q$, więc
$$\Autom(w) \in F_\Autom \Leftrightarrow \Autom'(w) \in F' \Leftrightarrow \mathcal{B}(w) \in
\breve Q \land \mathcal{B}(w) \in F' \Leftrightarrow \mathcal{B}(w) \in \breve F$$
\QEDB
\end{itemize}

\paragraph{Fakcik 2:}
Spróbujmy oszacować $|\breve Q|$. Dla $m = |Q|$ zachodzi:
\begin{itemize}
  \item $|Q'| = \bigsum_{i=0}^{m} \binom{m}{i}$
  \item $
    \begin{aligned}[t]
	  |\breve Q| &= 1 + \bigsum_{i=0}^{10} \binom{m}{i} =
	  1 + \bigsum_{i=0}^{10} \frac{m!}{i!(m-i)!} =
	  1 + \bigsum_{i=0}^{10} \frac{\prod_{k=0}^{i-1}m-k}{i!} \leq \\ &\leq
	  1 + \bigsum_{i=0}^{10} \prod_{k=0}^{i-1}m = 1 + \bigsum_{i=0}^{10} m^{i} \leq
	  1 + \bigsum_{i=0}^{10} m^{10} = 11m^{10} + 1 = O(m^{10})
    \end{aligned}
  $
  \item $|\breve Q| \leq O(m^{10}) = O(|Q|^{10})$
\end{itemize}

\clearpage

\paragraph{Algorytm 1:}
Automat $\mathcal{B}$ można wygenerować w czasie wielomianowym. 
\subparagraph{}
Przyjmijmy że mamy bezpośredni dostęp do zbiorów $Q, A, \delta$ przytoczonych w treści zadania.
Przyjmijmy również, że struktura zbioru udostępnia
\begin{itemize}
	\item konstruktor pustego zbioru o stałej złożoności czasowej
	\item procedury przeszukiwania zawartości oraz wstawiania elementów, obydwie o czasie
              liniowym względem rozmiaru przetrzymywanych danych
\end{itemize}

Takie złożoności czasowe są możliwe do osiągnięcia np. przy implementacji zbioru za pomocą
pojedynczo linkowanej listy (liczymy na to, że nie dotyczy nas ekstremalny, o ile w ogóle możliwy
przypadek, w którym ewentualna konwersja udostępnionej nam struktury do takiej struktury zbioru
nie byłaby możliwa do wykonania w wielomianowym czasie).


\paragraph{1.}
Pierwszym krokiem algorytmu bedzie wygenerowanie zbioru $\breve Q$.
Jego reprezentantem w naszym kodzie będzie $\texttt{QB}$. Z definicji ma on być zbiorem 
wszystkich podzbiorów $Q$ o mocy niewiększej niż 10 (plus śmietnik). 
\subparagraph{}
\begin{lstlisting}[escapeinside={(*}{*)}]
QB := {(*$\bot$*), {}}
elems := {{}}  # wszystkie (i-1)-elementowe podzbiory Q
temp := {}  # tu na podst. elems generujemy i-elem. podzb. Q

for (i := 1; i <= 10; i++):
    for e in elems:
	for q in (*$Q$*):
	    e.insert(q)
	    temp.insert(e)
    for t in temp:
	QB.insert(t)
    swap(temp, elems)
    temp := {}
\end{lstlisting}
\subparagraph{}
Za pomocą trywialnej indukcji można pokazać, że komentarze z linii 2-3 są niezmiennikami pętli z
linii 5 w momencie gdy program dochodzi do linii 10. Wobec tego
\texttt{temp}, \texttt{elems} $\subseteq \breve Q$, więc
$|\texttt{temp}|+|\texttt{elems}| \leq O(|Q|^{10})$.
Ponadto, gdy przez $\texttt{temp}_j$
oznaczymy zawartość struktury $\texttt{temp}$ w momencie gdy program jest w linii 10 
a wartość $\texttt{i}$ wynosi $j$, wówczas dzięki liniom 10-11 zachodzi
$$\begin{aligned}
	\texttt{QB} &= (\sum_{j=1}^{10} \texttt{temp}_j) \cup \{\bot, \varnothing\} 
	=\sum_{j=0}^{10} \{S \mid S \in P(Q) \land |S| = j\} \cup \{\bot\} = \\
	&= \{S \mid S \in Q' \land |S| \leq 10\} \cup \{\bot\} = \breve Q
\end{aligned}$$
 (a więc zgodnie z definicją).
\subparagraph{}
Powyższy kod składa się z trzech czasowo liniowych, zagnieżdżonych przejść po zbiorach, kolejno
$\mathbb{N}_{10}$, \texttt{elems} i $Q$, gdzie $|\texttt{elems}| \leq |\breve Q|
\cdot |Q| \leq O(|Q|^{11})$ \mbox{i $|\mathbb{N}_{10}| = O(1)$}. Z kolei czas każdej z podprocedur
\texttt{insert} jest ograniczony przez rozmiar struktur, na których jest wykonywana, a więc 
co najwyżej $O(|Q|^{10})$. Sumaryczny czas wykonania tego programu $T_1$ będzie więc rzędu
co najwyżej $O(|Q|^{11} \cdot 1 \cdot |Q| \cdot |Q|^{10}) \leq O(|Q|^{30})$

\paragraph{2.}
Wyliczenie $\texttt{FB} = \breve F$ jest trywialne.
\begin{lstlisting}[escapeinside={(*}{*)}]
FB := QB.filter(q => {
    for e in q:
	for f in (*$F_\Autom$*):
	    if f == e: return true
    return false
})
\end{lstlisting}
\subparagraph{}
Procedura \texttt{filter} jest możliwa do wyrażenia za pomocą fora i jednego bufora, kod zapisałem
w ten sposób tak, by zachować czytelność. Złożoność czasowa tej procedury jest identyczna ze
złożonością przejścia \texttt{for}em po strukturze \texttt{QB}.
\subparagraph{}
\texttt{FB} po przejściu algorytmu będzie zawierał te elementy $\breve Q$, które mają część wspólną
z $F_\Autom$ (bo istnieje element zawarty w obydwu zbiorach), więc $\texttt{FB} =
\{S \mid S \subseteq \texttt{QB} \land S \cap F_\Autom \neq \emptyset\} =
\{S \mid S \subseteq \breve Q \land S \cap F_\Autom \neq \emptyset\} =
\{S \mid S \subseteq Q' \land S \cap F_\Autom \neq \emptyset\} \cap \breve Q = F' \cap \breve Q =
\breve F$.
\subparagraph{}
Czas wykonania się powyższego kodu oczywiście jest wielomianowy z racji rozmiarów przeglądanych
struktur oraz metod na nich wykonywanych - możemy zastosować rozumowanie podobne do
przeprowadzonego wyżej.
Powinno nam wyjść $T_2 \leq O(|\texttt{QB}| \cdot 10 \cdot |F_\Autom|) \leq
O(|Q|^{10} \cdot |Q|) \leq O(|Q|^{11})$ (zauważamy że $\texttt{QB} = \breve Q$ oraz
$F_\Autom \subseteq Q$).

\paragraph{3.}
W końcu postaramy się wyliczyć funkcję przejścia $\breve \delta$. Jest to ostatnia rzecz potrzebna
nam do tego, żeby automat zadziałał - pozostałe rzeczy, czyli alfabet $A$ oraz stan początkowy 
$q_0 = I_\Autom$ można łatwo wywnioskować z danego nam na początku automatu $\Autom$.
\subparagraph{}
Do spamiętania wartości funkcji $\breve \delta$ możemy zastosować np. listową implementację
słownika \texttt{dict(p $\in$ QB -> dict($\alpha \in A$ -> q $\in$ QB))}. Sprawdzanie wartości tak
zapisanej funkcji, a także ustawienie nowej wartości dla pewnej pary argumentów da się zrealizować
w czasie liniowym względem rozmiaru danych, czyli w czasie $O(|\texttt{QB}| \cdot |A|) =
O(|A|\cdot|Q|^{10})$.

\begin{lstlisting}[escapeinside={(*}{*)}]
DB := dict()

for s in QB:
    DB[s] := dict()
    for a in (*$A$*):
	temp := {}  # wszystkie stany osiagalne
                    # z podstanow s po wczytaniu a
        for q in s:
	    for el in (*$\delta$*):
		if el[0] == q && el[1] == a:
		    temp.insert(el[2])
        if s == (*$\bot$*) || temp.size > 10:
	    temp := (*$\bot$*)
	DB[s][a] := temp

	    
\end{lstlisting}

Zachowane są niezmienniki pętli:
\begin{itemize}
  \item 9 - po zakończeniu pętli \texttt{temp} zawiera wszystkie stany osiągalne przez wczytanie \texttt{a}
do $\Autom$ w stanie \texttt{q}
  \item 8 - po wykonaniu spełniony jest komentarz z linii 6-7
  \item 5 - \texttt{DB[s][a]} zawiera $\breve \delta(\texttt{s}, \texttt{a})$
  \item 3 - $\forall_{s \in \breve Q}\forall_{a \in A} \texttt{DB[}s\texttt{][}a\texttt{]} =
	    \breve\delta(s, a)$
\end{itemize}

Trzy zagnieżdżone pętle przeglądają struktury poszczególnych zbiorów w łącznym czasie
$O(|\texttt{QB}| \cdot |A| \cdot 10 \cdot |\delta|) = O(|Q|^{10} \cdot |A| \cdot |\delta|)$.
Z kolei rozmiar \texttt{temp}, a więc i czas wykonywania się najbardziej kosztownej procedury
(linia 11) jest na pewno ograniczony przez $O(3 + |\breve Q|)$, więc $T_3 \leq
O(|Q|^{30} \cdot |A| \cdot |\delta|)$.

\paragraph{}
Koniec końców czas działania naszego algorytmu $T = \sum_{i \in \{1, 2, 3\}}T_i \leq
O(|Q|^{30})+O(|Q|^{11})+O(|Q|^{30} \cdot |A| \cdot |\delta|) =
O(|Q|^{20}(|Q|^{10} + |A| \cdot |\delta|) \leq O(h^{20}(h^{10} + h \cdot h) = O(h^{30})$ gdzie
$h = |Q| + |A| + |\delta|$. Istnieje więc algorytm generujący automat deterministyczny z automatu
$10$-niedeterministycznego, którego czas jest ograniczony przez wielomian zależny od czynników
$h$, co było do pokazania w tym zadaniu.
\subparagraph{}
Oczywiście wygenerowanie tego automatu jest możliwe przy dużo mniejszej złożoności czasowej,
powyższy algorytm można dużo lepiej zoptymalizować, jednak nie było to celem zadania. Starałem się
dobierać jak najprostsze struktury, a moje oszacowania złożoności czasowej są bardzo zgrubne.
\subparagraph{}
W niniejszym opisie algorytmu mogłem czasem przywiązywać mniejszą uwagę do problemu złożoności
czasowej głębokiego kopiowania struktur (m.in. nie skupiałem się na tym, kiedy płytka kopia mogłaby
przyspieszyć program - patrz akapit wyżej). Poszczególne z tych procedur są jednak wykonywane na
strukturach o rozmiarze, który da się ograniczyć przez wielomianową kombinację którychś z
parametrów $|Q|, |A|, |\delta|$, a głebokie kopiowanie tych struktur da sie wykonać w czasie
wielomianowym ze względu na rozmiar przetrzymywanych w nich danych. Suma oraz iloczyn dwóch 
wielomianów daje wielomian, dzięki czemu to zagadnienie również nie powinno mieć znaczenia dla
faktu realizowalności tego zadania w czasie wielomianowym. 

\label{LastPage}

\end{document}
